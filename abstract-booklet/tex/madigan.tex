
\addcontentsline{toc}{subsection}{Dr. David Madigan}
\begin{center}
\Large Honest Learning for the Healthcare System: \\ Large-scale
Evidence from Real-world Data \\[1em]
\end{center}
\normalsize \textbf{Dr. David Madigan}, Columbia University \\[.75em]
\footnotesize(joint work with Martijn J. Schuemie, Patrick B. Ryan, George
Hripcsak, and Marc A. Suchard) \\[.5em]

\normalsize In practice, our learning healthcare system relies primarily on
observational studies generating one effect estimate at a time using
customized study designs with unknown operating characteristics and
publishing---or not---one estimate at a time. When we investigate
the distribution of estimates that this process has produced, we see
clear evidence of its shortcomings, including an over-abundance of
estimates where the confidence interval does not include one
(i.e. statistically significant effects) and strong indicators of
publication bias. In essence, published observational research
represents unabashed data fishing. We propose a standardized process
for performing observational research that can be evaluated,
calibrated and applied at scale to generate a more reliable and
complete evidence base than previously possible, fostering a truly
learning healthcare system. We demonstrate this new paradigm by
generating evidence about all pairwise comparisons of treatments for
depression for a relevant set of health outcomes using four large US
insurance claims databases. In total, we estimate 17,718 hazard
ratios, each using a comparative effectiveness study design and
propensity score stratification on par with current state-of-the-art,
albeit one-off, observational studies. Moreover, the process enables
us to employ negative and positive controls to evaluate and calibrate
estimates ensuring, for example, that the 95\% confidence interval
includes the true effect size approximately 95\% of time. The result
set consistently reflects current established knowledge where known,
and its distribution shows no evidence of the faults of the current
process. Doctors, regulators, and other medical decision makers can
potentially improve patient-care by making well-informed decisions
based on this evidence, and every treatment a patient receives becomes
the basis for further evidence.

\hrulefill

\small David Madigan is the Executive Vice-President for Arts \& Sciences,
Dean of the Faculty, and Professor of Statistics at Columbia
University in the City of New York. He previously served as Chair of
the Department of Statistics at Columbia University (2008--2013), Dean,
Physical and Mathematical Sciences, Rutgers University (2005--2007),
Director, Institute of Biostatistics, Rutgers University (2003--2004),
and Professor, Department of Statistics, Rutgers University
(2001-2007).  He received his bachelor's degree in Mathematical
Sciences (1984, First Class Honours, Gold Medal) and a Ph.D. in
Statistics (1990), both from Trinity College Dublin.

Dr. Madigan has over 160 publications in such areas as Bayesian statistics,
text mining, Monte Carlo methods, pharmacovigilance and probabilistic
graphical models. In recent years he has focused on statistical
methodology for generating reliable evidence from large-scale
healthcare data. From 2011 to 2014 he was a member of the FDA's
Drug Safety and Risk Management Advisory Committee.

Dr. Madigan is a fellow of the American Association of the Advancement
of Science (AAAS), the Institute of Mathematical Statistics (IMS) and
the American Statistical Association (ASA), and an elected member of
the International Statistical Institute (ISI).  He served as
Editor-in-Chief of Statistical Science (2008--2010) and Statistical
Analysis and Data Mining, the ASA Data Science Journal (2013--2015).

\normalsize

