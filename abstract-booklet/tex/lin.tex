
\addcontentsline{toc}{subsection}{Dr. Xihong Lin}
\begin{center}
\Large Hypothesis Testing for Weak and Sparse Alternatives With
Applications to Whole Genome Data \\[1em]
\end{center}
\normalsize \textbf{Dr. Xihong Lin}, Harvard University \\[.5em]

Massive genetic and genomic data generated using array and sequencing
technology present many exciting opportunities as well as challenges
in data analysis and result interpretation, e.g., how to develop
effective strategies for signal detection using massive genetic and
genomic data when signals are weak and sparse. In this talk, I will
discuss hypothesis testing for sparse alternatives in analysis of
high-dimensional data motivated by gene, pathway/network based
analysis in genome-wide association studies using arrays and
sequencing data. I will focus on signal detection when signals are
weak and sparse, which is the case in genetic and genomic association
studies. I will discuss hypothesis testing for signal detection using
variable selection based penalized likelihood based methods, the
Generalized Higher Criticism (GHC) test, and the Generalized
Berk-Jones test, and the robust omnibus test. I will discuss the
challenges in statistical inference in the presence of both
between-observation correlation and signal sparsity. The results are
illustrated using data from genome-wide association studies and
sequencing studies.

\hrulefill

\small Xihong Lin is Chair and Henry Pickering Walcott Professor of
Department of Biostatistics and Coordinating Director of the Program
of Quantitative Genomics at the Harvard T. H. Chan School of Public
Health, and Professor of Statistics of the Faculty of Art and Science
of Harvard University.

Dr. Lin's research interests lie in development
and application of statistical and computational methods for analysis
of massive genetic and genomic, epidemiological, environmental, and
medical data. She currently works on whole genome sequencing
association studies, genes and environment, analysis of integrated
data, and statistical and computational methods for massive health
science data.

Dr. Lin received the 2002 Mortimer Spiegelman Award from
the American Public Health Association and the 2006 COPSS Presidents'
Award. She is an elected fellow of ASA, IMS, and ISI. Dr. Lin received
the MERIT Award (R37) (2007--2015), and the Outstanding
Investigator Award (OIA) (R35) (2015--2022) from the National
Cancer Institute. She is the contacting PI of the Program Project
(PO1) on Statistical Informatics in Cancer Research, the Analysis
Center of the Genome Sequencing Program of the National Human Genome
Research Institute, and the T32 training grant on interdisciplinary
training in statistical genetics and computational biology. Dr. Lin
was the former Chair of the COPSS (2010--2012) and a former
member of the Committee of Applied and Theoretical Statistics (CATS)
of the National Academy of Science. She is the former Chair of the new
ASA Section of Statistical Genetics and Genomics. She was the former
Coordinating Editor of Biometrics and the founding co-editor of
Statistics in Biosciences, and is currently the Associate Editor of
Journal of the American Statistical Association. She has served on a
large number of statistical society committees, and NIH and NSF review
panels. \\[2em]

\normalsize

